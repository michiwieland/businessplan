% !TeX spellcheck = de_CH
\documentclass[
a4paper,
oneside,
12pt,
fleqn,
headsepline,
toc=listofnumbered, 
bibliography=totocnumbered]{scrartcl}

% deutsche Trennmuster etc.
\usepackage[T1]{fontenc}
\usepackage[utf8]{inputenc}
\usepackage[english, ngerman]{babel} % \selectlanguage{english} if  needed
\usepackage{lmodern} % use modern latin fonts

% Custom commands
\newcommand{\AUTHOR}{Michael Wieland}
\newcommand{\SECONDAUTHOR}{Fabian Hauser}
\newcommand{\THIRDAUTHOR}{Murièle Trentini}
\newcommand{\FOURTHAUTHOR}{Patrik Scherler}
\newcommand{\INSTITUTE}{Hochschule für Technik Rapperswil}
\newcommand{\LECTURER}{Prof. Dr. Jürg Stadelwieser}
\newcommand{\GITHUB}{https://github.com/michiwieland/businessplan}

% Jede Überschrift 1 auf neuer Seite
\let\stdsection\section
\renewcommand\section{\clearpage\stdsection}

% Multiple Authors
\usepackage{authblk}

% Include external pdf
\usepackage{pdfpages}

% Layout / Seitenränder
\usepackage{geometry}

% Inhaltsverzeichnis
\usepackage{makeidx} 
\makeindex

\usepackage{url}
\usepackage[pdfborder={0 0 0}]{hyperref}
\usepackage[all]{hypcap}
\usepackage{hyperxmp} % for license metadata

% Glossar und Abkürzungsverzeichnis
\usepackage[acronym,toc,nopostdot]{glossaries}
\glossarystyle{altlisthypergroup}
\usepackage{xparse}
\DeclareDocumentCommand{\newdualentry}{ O{} O{} m m m m } {
	\newglossaryentry{gls-#3}{name={#5},text={#5\glsadd{#3}},
		description={#6},#1
	}
	\makeglossaries
	\newacronym[see={[Siehe:]{gls-#3}},#2]{#3}{#4}{#5\glsadd{gls-#3}}
}
\makeglossaries

% Mathematik
\usepackage{amsmath}
\usepackage{amssymb}
\usepackage{amsfonts}
\usepackage{enumitem}

% Images
\usepackage{graphicx}
\graphicspath{{images/}} % default paths

%figures
\usepackage{tikz}
\usetikzlibrary{shapes.geometric}


% Boxes
\usepackage{fancybox}

%Tables
\usepackage{tabu}
\usepackage{booktabs} % toprule, midrule, bottomrule
\usepackage{array} % for matrix tables

% Multi Columns
\usepackage{multicol}

% Header and footer
\usepackage[automark]{scrlayer-scrpage}
\setkomafont{pagehead}{\normalfont}
\setkomafont{pagefoot}{\normalfont}
\automark*{section}
\clearpairofpagestyles
\ihead{\headmark}
\ohead{\TITLE}
\cfoot{\pagemark}
\setlength{\headsep}{30pt}


%Defining the Layer
\makeatletter
\newlength{\topheight}
\setlength{\topheight}{\sls@topmargin}
\addtolength{\topheight}{\headheight}
\DeclareLayer[
background,
contents={%
	\color{\chaptercolor}%
	\rule{\paperwidth}{\topheight}%
}%
]{scrheadings.head.background}
\makeatother

%Adding the Layer to the pagestyles
\AddLayersAtBeginOfPageStyle{scrheadings}{%
	scrheadings.head.background,%
}
\AddLayersAtBeginOfPageStyle{plain.scrheadings}{%
	scrheadings.head.background}

\usepackage{etoolbox}
\newcommand{\chaptercolor}{PrimaryColor}


% Pseudocode
\usepackage{algorithm}
\usepackage{algorithmic}

% Code Listings
\usepackage{listings}
\usepackage{color}
\usepackage{beramono}

\definecolor{DarkPurple}{rgb}{0.4, 0.1, 0.4}
\definecolor{DarkCyan}{rgb}{0.0, 0.5, 0.4}
\definecolor{LightLime}{rgb}{0.3, 0.5, 0.4}
\definecolor{Blue}{rgb}{0.0, 0.0, 1.0}

% Page colors
\definecolor{PrimaryColor}{HTML}{CDDC39}
\definecolor{SecondaryColor}{HTML}{374046}

\usepackage{afterpage}
\usepackage{xcolor}
\usepackage{sectsty}

\color{SecondaryColor}

\lstdefinestyle{eclipse-style}{
	language=Java,
	columns=flexible,
	showstringspaces=false,
	basicstyle=\footnotesize\ttfamily, 
	keywordstyle=\bfseries\color{DarkPurple},
	commentstyle=\color{LightLime},
	stringstyle=\color{Blue}, 
	escapeinside={£}{£}, % latex scope within code
	morekeywords={length},
	numbers=left,
	numberstyle=\tiny\color{black},
	frame=single,
}
\lstset{style=eclipse-style}


%risk-rating
\newcommand\risk[2]{
	\begin{tikzpicture}
	\draw [thick, |->] (0,2) -- (#2,2);
	\draw [fill=PrimaryColor, thick] (#1,2) circle [radius=0.2];
	\end{tikzpicture}
}

% Theorems \begin{mytheo}{title}{label}
\usepackage{tcolorbox}
\tcbuselibrary{theorems}
\newtcbtheorem[number within=section]{definiton}{Definition}%
{fonttitle=\bfseries}{def}
\newtcbtheorem[number within=section]{remember}{Merke}%
{fonttitle=\bfseries}{rem}
\newtcbtheorem[number within=section]{hint}{Hinweis}%
{fonttitle=\bfseries}{hnt}

% Dokumentinformationen
\newcommand{\SUBJECT}{Businessplan}
\newcommand{\TITLE}{GitFit}

% pdf metadata
\hypersetup{
	pdfauthor={\AUTHOR},
	pdftitle={\SUBJECT \TITLE}
}

\begin{document}

% German \and
\renewcommand\Authands{ und }
% Front page
\title{\TITLE}
\subject{\SUBJECT}
\author{\SECONDAUTHOR}
\author{\THIRDAUTHOR}
\author{\FOURTHAUTHOR}
\author{\AUTHOR}
\affil{\INSTITUTE}
\affil{\LECTURER}
\date{\today}
\maketitle

%TODO add front image
\begin{center}
%	\includegraphics[width=0.7\linewidth]{images/front}
\end{center}

\vfill

% Github
\paragraph{Github:} \url{\GITHUB}

% Table of contents
\tableofcontents

% Glossar and acronyms (if included \loadglsentries{glossar})
\printglossary[type=\acronymtype]
\printglossary
\glsaddall


%TODO Total: 20 Seiten -> Es muss ersichtlich sein, was das Produkt bringt.
%TODO Sturktur nach PWC -> Siehe PDF auf Skripte Server

\section{Abstract}


% TODO Wir bieten flexibiltät, effizienz, komfort


\section{Das Businessmodell}
%TODO Import Canvas \includepdf[pages={1},landscape=true]{appendix/schemes/datacenter.pdf}

\section{Mission, Vision und Strategie: die Zukunft}
\subsection{Mission}
Unsere Aufgabe ist es, vielen Menschen ein völlig neues Erlebnis für ein persönliches Training im Fitnesscenter zu bieten. Wir tun dies, indem wir eine App anbieten, welche die Fitnesslandschaft revolutioniert. Dies zu Preisen, die sich auch kleine Fitnesscenter leisten können.

\subsection{Vision}
Schweizer Fitnesscenter setzen aktuell vorzugsweise auf Papier und Bleistift für die Trainingsplanung ihrer Kunden. Dies ist nicht mehr zeitgemäss und bedarf einer Generalüberholung. Am Puls der Zeit zu sein, bedeutet für ein Fitnesscenter, einen modernen und attraktiven Dienstleister für seine Kunden darzustellen. Die Digitalisierung bietet unendlich viele Möglichkeiten, die es zu nutzen gilt. \\
Wir wollen eine App kreieren, welche die Fitnessbranche wieder auf den aktuellen Stand der Technik katapultiert. Die App bietet aufschlussreiche Statistiken sowie hilfreiche Tipps zur Übung, wenn der Trainer gerade nicht in der Nähe ist.

\subsection{Strategie}
Unser Angebot richtet sich in erster Linie an grosse Fitnesscenter-Ketten, da wir glauben, dass diese an einem übergreifenden Wissensaustausch der Filialen interessiert sind. Im Rahmen der Entwicklung kümmern wir uns in einer ersten Phase um das Anreichern von Stammdaten mit auserwählten Partner-Center. Die Daten sollen als solides und realitätsnahes Fundament für die weitere Entwicklung dienen. In einer zweiten Phase wird dann die App für den Endkunden entwickelt, die ein möglichst komfortables und persönliches Trainingserlebnis vermitteln soll. 

\subsubsection{Wettbewerbsstrategie}
Der Schweizer Fitnessmark boomt und trotzdem ist keine Digitalisierung in der Branche erkennbar. Ähnliche Projekte finden im Ausland bereits Anklang, jedoch gibt es im Inland keine vergleichbare Dienstleistung. Grund genug, diese Lücke zu füllen. Die Schwierigkeit in dem Projekt liegt deshalb weniger in der Positionierung gegenüber ausländischen Dienstleister, sondern vielmehr die heimischen Fitnesscenter mit der neuen Technologie vertraut zu machen.


\section{Produkte und Dienstleistungen: die Marktleistung}

\subsection{Die App}
\begin{figure}[h]
\centering
\includegraphics[width=0.5\linewidth]{images/app}
\caption{Wireframe der App}
\label{fig:app}
\end{figure}


\section{Markt und Kunden: das Zielgebiet}
%So viele Kunden gibt's, und so viele würden das kaufen. Gibt es einen Markt dafür?


%TODO: Kansroosa
Unsere Zielgruppe

%TODO: Slide 10, Part 5

\subsection{Marktübersicht}

\subsubsection{Marktkapazität / Marktsegmentierung}\label{sec:marktkapazitat}

\paragraph{Fitnesscenter in der Schweiz}
Gemäss einer Studie des BASPO Statistik aus dem Jahre 2014 sind 16\% der Schweizerinnen und Schweizer in einem privaten Fitnesscenter angemeldet \cite{schweizer+fitness}. Damit ist das betreiben eines Fitnesscenters eine lukrative Angelegenheit; dadurch gibt es inzwischen zahlreiche Fitnesscenter, bei verschiedenen Quellen wird die Anzahl auf 700-800 Fitnesscentern im Jahre 2013 geschätzt \cite{fitness-studios+1+milliarde}\cite{fitness+tribune}.

Die meisten Fitnesscenter setzten digitale Mittel lediglich zur Verwaltung des Kundenstammes und Administration ein; nach unseren Recherchen sind weitere Verwaltungsapplikationen bisher nicht im Einsatz.

\paragraph{Grössere Fitnesscenter}
In der Schweiz gibt es einige grössere Anbieter, welche eine Reihe von Fitnesscenter betreiben. Diese sind interessante Abnehmer für unsere Applikation, da diese viele Kunden, und damit entsprechende Datenmengen zu verwalten haben.

Die grössten schweizer Fitnesscenter sind\cite{fitness+tribune}:
\begin{itemize}
	\item \url{http://mfit.ch/}
	\item \url{http://www.activfitness.ch/}
	\item \url{http://fitnessconnection.ch/}
	\item \url{http://www.fitnesspark.ch/}
	\item \url{http://holmesplace.ch/de/}
	\item \url{http://www.kieser-training.com/}
	\item \url{http://www.swiss-training.com/}
	\item \url{http://tc-training.ch/}
\end{itemize}

\paragraph{Medizinische Fitnesscenter / Physiotherapie}

Neben den \emph{Fitness}centern gibt es auch einige Physiotherapie-Center. Diese sprechen in erster Linie Kunden mehr über einen Medizinischen Hintergrund und nicht wie die meisten Fitnesscenter mit der allgemeinen ''Fitness'' an.

Für die Physiotherapie sind meistens spezielle Physiogeräte nötig; die Kundenstämme und Pläne werden üblicherweise über medizinische Verordnungen gesteuert. Dies impliziert auch erweiterte Datenschutzansprüche (Arztgeheimnis etc.), welche eher aufwändig umzusetzen sind.

\subsubsection{Marktakteure}
In der Schweiz gibt es bisher keine Anbieter von Apps, welche sich direkt an Fitnesscenter wenden. Wir haben uns das schweizer App ''myClubs'' angeschaut, das sich direkt an professionelle Sportler oder Amateursportler wendet.

In Deutschland gibt es einige wenige Anbieter von Apps für den Fitnessgebrauch; die meisten richten sich aber an Sportler und gehen nicht auf Fitnesspläne oder interaktion mit Fitnessgeräten ein. Daneben betreiben ein paar Fitnessketten kleinere Eigenentwicklungen, welche nicht weiter verbreitet sind. Die Firma \emph{eGym} ist ein potentieller direkter Konkurrent.

Mehr Informationen zur potentiellen Konkurrenz sind im Kapitel \ref{sec:konkurrenz-die-mitbewerber} zu finden.


\subsubsection{Erfolgsfaktoren im Markt}

Ein wesentliches Problem von Fitnesscentern ist das halten ihrer Kunden, da inzwischen eine ziemlich ausgebreitete Konkurrenz existiert. Aus diesem Grund werden die Fitnesscenter die App wählen, welche die grösste Kundenbindung ermöglicht. Dabei spielen Design, Qualität und vor allem der gute Ruf eine Rolle.

\subsubsection{Marktenwicklung}
Der Markt für unsere App ist in den letzten Jahren stark gewachsen \cite{fitness-studios+1+milliarde}\cite{fitness+tribune}, allerdings ist es schwierig, genaue Zahlen zu finden. In den nächsten Jahren wird das Wachstum aufgrund der bestehenden hohen Zahlen etwas zurückgehen. Zunehmend gibt es auch grössere Fitnessketten (siehe Details im Abschnitt \ref{sec:marktkapazitat}).

Aufgrund der beschränkten Zahl Fitnesscenter ist es wichtig, unser App gut zu Profilieren und relativ schnell einen hohen Bekanntheitsgrad zu erlangen.

\subsection{Chancen und Risiken / Nachfrage \& Charakteristika}

Da in der Schweiz erst wenige Fitness-Center im digitalen Zeitalter angekommen sind, gibt es hier eine grosse Chance, Fuss auf dem Markt zu fassen. Einmal etabliert ist es für Konkurrenten schwierig, Fuss zu fassen.

Das grösste Risiko geht von dieser Seite her von der deutschen Konkurrenz aus, da diese aufgrund der finanziellen Gegebenheiten das Produkt zu deutlich günstigeren Preisen anbieten können.


\subsection{Eigene Marktstellung}
%http://www.fitness-expo.ch/

%TODO
Als Startup können wir unsere Firma mit einer neuen Marke auf dem Markt bewegen. Als Startup aus Hochschulkreisen ist einiges an technischem Know-How vorhanden. Bisher existieren aber keine über die Bedarfsanalyse herausgehende Geschäftskontakte in diesem Bereich.
%TODO: Braucht diese Sektion noch mehr?

% Welche Produkte und Dienstleistungen werden an welche Zielkunden durch welche Vertriebskanäle auf welchen geografischen Märkten verkauft?
% Welche speziellen Marktsegmente bearbeiten wir?
% Welches ist unser aktueller Marktanteil, unser Einfluss auf dem Markt?
% Welches sind unsere Pläne betreffend Exportmarkt?
% Wie gross sind die einzelnen Umsatz- bzw. Gewinnanteile pro Marktleistung?
% Welche Besonderheiten gibt es pro Produkt/Marktsegment?


\section{Konkurrenz: die Mitbewerber}\label{sec:konkurrenz-die-mitbewerber}

Auf dem schweizer Markt ist die Konkurrenz noch eher klein. Erst wenige Fitnesscenter haben sich selber oder wurden von einem Dienstleister digitalisiert. Jedoch gibt es einige Anbieter im nahen Ausland, für welche die Schweiz einen attraktiven Mark bietet. Beide Gruppen wollen wir als potentielle Konkurrenten analysieren.
\subsection{Konkurrenzunternehmen}
\subsubsection{myClubs}\hfill \\
myClubs\cite{myclubs} ist ein schweizer Anbieter einer Fitness App. Ihr Ziel ist es, einem Sportler eine flexible Möglichkeit zu bieten Sport zu treiben. Dabei legen sie ihren Fokus jedoch nicht darauf Fitnesscenter attraktiver zu gestalten oder zu digitalisieren, sondern dem Sportler ein breites Sportangebot in einem einzigen Abo zu bieten. Im Gegensatz zu unserer Strategie, sind die Endkunden von myClub also Sportler, nicht Fitnesscenter.
Das myClubs Angebot umfasst:
\begin{itemize}
	\item Auswahl an verschiedenen Sportarten und -kursen
	\item Auswahl an verschiedenen Partner-Anbietern
	\item Zum Fixpreis im Abosystem
\end{itemize}
Wir heben uns vom Konkurrenten myClub ab, da wir unseren Fokus auf die Verbesserung des Kraft- und Ausdauertrainings in einem Fitnesscenter legen.
\paragraph{Risikoeinschätzung} \qquad \risk{4}{5}
\subsubsection{eGym}\hfill \\
Der deutsche Anbieter eGym\cite{egym} verfolgt sehr ähnliche Ziele wie wir. Sie bieten ihren Partnerfitnesscentern eine App mit folgenden Kernfunktionen:
\begin{itemize}
	\item Zugriff auf den Trainingsplan des Fitnesstrainers
	\item Dokumentation von Trainingseinheiten
	\item Kontrolle über Trainingserfolg und Vergleich mit Kollegen
\end{itemize}
eGym wäre ein starker Konkurrent, falls er in den schweizer Markt expandieren würde.
\paragraph{Risikoeinschätzung} \qquad \risk{3}{5}
\subsubsection{Technogym}\hfill \\
Technogym\cite{technogym}ist ein Anbieter aus der Niederlande. Sie bewegen sich im selben Marktsegment wie wir. Sie verkaufen ihre Fitnessapp sowie angebundene Fitnessgeräte an ihre Fitnesscenter-Kunden.
Ihre Produktpalette umfasst:
\begin{itemize}
	\item Fitnessgeräte mit digitalen Funktionen
	\item vom Trainer erstellte Fitnesspläne mit der Prescribe-App
\end{itemize}
Dabei legen sie ihren Fokus aber vor allem auf den Vertrieb von Fitnessgeräten, wobei wir uns auf die Digitalisierung und das Einbinden der App konzentrieren.
\paragraph{Risikoeinschätzung} \qquad \risk{2}{5}
\subsubsection{Freeletics}\hfill \\
Der deutsche Dienstleister Freeletics\cite{freeletics} entwickelt eine eigene Fitness-App und betreibt eine Fitnesscenter-Kette.
Folgendes Angebot bietet Freeletics:
\begin{itemize}
	\item App mit Übungsanleitungen und Fortschrittstracking
	\item Ihre eigene Fitnesscenter-Kette
\end{itemize}
Da Freeletics ihre eigene Fitnesscenter-Kette betreibt, wären sie nur dann ein ernsthafter Konkurrent, wenn sie in die Schweiz expandieren und hier ansässige Fitnesscenter vom Markt vertreiben würden.
\paragraph{Risikoeinschätzung} \qquad \risk{2}{5}
\subsection{Konkurrenzanalyse}
\paragraph{Wie arbeitet die Konkurrenz?} \hfill \\
Passend zum Marktsegment haben alle Konkurrenten eine starke Internet-Präsenz mit ansprechenden, modernen Webseiten.
\paragraph{Wieso sind wir unserer Konkurrenz überlegen?}\hfill \\
\begin{figure}[h]
\centering
\includegraphics[width=0.9\linewidth]{images/konkurrenz}
\caption{Konkurrenzanalyse}
\label{fig:konkurrenz}
\end{figure}
\paragraph{Konsequenzen}\hfill \\
Aus der Konkurrenzanalyse ergab sich, dass wir uns auf folgende Punkte konzentrieren wollen:
\begin{itemize}
	\item Hoher Preis, dafür sehr gute schweizer Qualität
	\item Modernes Auftreten und Design mit einfacher Handhabung
\end{itemize}

\section{Marketing: der Weg zum Markt}

\subsection{Marketingziele}
Um unseren Zielmark zu erreichen, haben wir uns quantitave Ziele gesetzt:
\begin{itemize}
	\item Bis Februar 2017 soll ein Prototyp der App vorgestellt werden können. Dieser ist zwingend um neue Kunden von unserer Idee und den damit verbundenen Vorteilen zu überzeugen.
	\item Bis Mai 2017 ist ein starker Partner gefunden, der im idealfall über mehrere Fitnesscenter in unterschiedlichen Regionen verfügt.
	\item Bis Ende 2017 ist eine praxisnahe, moderne Trainingsapp entwicket. Die App soll von möglichst vielen Kunden Inputs profitieren und diese benutzerfreundlich digital Abbilden.
	\item Bis Sommer 2018 können mindestens 10 weitere Fintnesscenter mit GitFit ausgerüstet werden.
\end{itemize}

\subsection{Marktpositionierung}
Wir setzen primär auf den Schweizer Mark, wobei wir uns darauf konzentrieren, eine grosse Fitnesscenter-Kette als Partner zu gewinnen. Da die Branche von wenigen ''Big Player'' dominiert wird,  wird sich deren Einsatz von GitFit bei den Konkurrenzlinien und kleineren Fitnesscenter herumsprechen. Fitnesscenter die noch auf herkömmliche Arbeitsmethoden setzen, werden als nicht mehr zeitgemäss betrachtet und so zu einer Digitalisierung gezwungen. Wir unterscheiden zwischen zwei Typen von Fitnesscenter, die als Kunden in Frage kommen.

\paragraph{Fitnesscenter Ketten} \hfill \\
Die grossen Fitnesscenter-Ketten sind besonders am übergreifenden Wissensaustausch ihrer regionalen Niederlassungen interessiert. So können Trainingspläne von einem Trainer Komitee erstellt werden und national in den verschiedenen Fitnesscenter auf ihre Praktikabilität überprüft werden. Entsprechende Anpassungen wirken sich auf alle beteiligten Fitnesscenter aus. Ebenfalls herrscht eine grosse Konkurrenz zwischen den einzelnen Ketten. Sich abzuheben scheint immer wie schwieriger und deshalb ist GitFit ein willkommenes Produkt zur eigenen Hervorhebung im Markt. 

\paragraph{Kleinere private Fitnesscenter} \hfill \\
Die kleineren, meist privat geführten Fitnesscenter, sind an einer kostengünstigen Lösung interessiert, um nicht an Attraktivität gegenüber den ''Big Player'' zu verlieren und Schritt zu halten. 

\subsection{Preispolitik}
\subsubsection{Preisfindung}
GitFit wird als Abomodell angeboten, wobei die Abokosten monatlich entrichtet werden müssen. Dies erlaubt es bei insolventen Kunden schnell und unkompliziert die Partnerschaft aufzulösen. Die Kosten werden an der Anzahl an GitFit Accounts in einem Fitnesscenter gemessen. Somit korreliert der Produtkpreis mit der Grösse des Fitnesscenter, weshalb sich auch kleinere Anbieter das Produkt leisten können.

\paragraph{Monatliche Kosten} \hfill
\begin{table}[h]
	\centering
	\begin{tabu} to \linewidth {l r}
		\toprule 
		Beschreibung & Monatliche Abokosten (CHF) \\
		\midrule
		bis zu 200 Kunden & 7.- \\
		ab 200 Kunden & 5.- \\
		\bottomrule 
	\end{tabu} 
	\caption{Preisliste}
\end{table}

\paragraph{Einmalige Fixkosten} \hfill
\begin{table}[h]
	\centering
	\begin{tabu} to \linewidth {l r}
		\toprule 
		Beschreibung & Monatliche Abokosten (CHF) \\
		\midrule
		Sportlerapp & gratis \\
		Trainerapp & gratis \\
		iPad Air 2 für den Trainer (optional) & 400.- \\
		Zugang zu praxiserprobten Trainingspläne (einmalig) & 2750.- \\
		Austattung der Fitnessgeräte mit QR Codes (einmalig) & 500.-  \\
		\midrule
		\textbf{Total (komplettes Paket)} & \textbf{3650.-} \\
		\bottomrule 
	\end{tabu} 
	\caption{Einmalige Fixkosten}
\end{table}

\subsubsection{Preisdifferenzierung}
Den Fitnesscenter wird eine Plattform geboten, über welche sie die komplette Trainingsplanung selbständig durchführen können. Dank einer intuitiven Oberfläche entsteht kein Mehraufwand gegenüber der herkömmlichen Lösung mittels Stift und Papier. Der Mehrwert 

%TODO wie können wir günster sein wie unsere Konkurrenz? Vergleich mit Konkurenz absatz

\subsubsection{Rabatte}
Die App basiert auf Praxis erprobten Stammdaten die in einer ersten Phasen mit auserwählten Partnerorganisation aufgebaut werden sollen. Teilnehmende Fitnesscenter profitieren von einer vergünstigten Dienstleistung im ersten Jahr sowie dem Vorteil einer aktiven Mitgestaltung des finalen Produktes. 

\begin{table}[h]
	\centering
	\begin{tabu} to \linewidth {l r}
		\toprule 
		Beschreibung & Fixe Abokosten im ersten Jahr (CHF) \\
		\midrule
		Pro Kunde & 4.50 \\
		\bottomrule 
	\end{tabu} 
	\caption{Preisliste}
\end{table}

\subsection{Distribution}
Gerade am Anfang des Produkt soll eine enge Bindung zwischen der Entwicklung und den beteiligten Fitnesscenter bestehen. Für die Bekanntmachung unserer Dienstleistung setzen wir auf verschiedenen Distributionspfade. 

\subsubsection{Produktpräsentationen}
Als kleines Startup gehen wir persönlich bei auserwählten Fitnesscenter vorbei und stellen den Nutzen unserer Plattform im persönlichen Gespräch vor. Wir suchen nach einem Partner, der uns auch auf persönlicher und nicht nur finanzieller Ebene entspricht, und die Ideologie mit uns teilt. Nach einer kurzen, informativen Präsentation hat der potentielle Kunde Zeit, das Produkt live zu testen und erleben. Dabei legen wir darauf Wert, dass die Vorteile für das Fitnesscenter und nicht für den Endkunden hervorgehoben werden. Insbesordere die Vorteile der statistischen Auswertung müssen dem potentiellen Kuden nach der Präsentation klar sein.

\subsubsection{Internet} \hfill \\
Mit einer interaktivem und zeitgemässen Webauftritt wollen wir unseren Kunden das Produkt näher bringen. Neue Kunden können sich mit wenigen Klicks für eine Testphase registrierern, worauf der Schritt zum Abo nur noch ein kleiner ist. Man registriert sich auf der Webseite und bindet den Testaccount an das Fitnesscenter seines Vertrauens. Es soll auch möglich sein, einen GitFit Account zu erstellen ohne bei einem Fitnesscenter angemolden zu sein. In diesem Fall wird das Fintesscenter in der Region über einen interessierten Kunden informiert. Da die primäre Zielgruppe die Fitnesscenter an sich sind, soll die Webseite auch für diese Zielgruppe interessante Fakten zur Verfügung stellen. So sollen einerseits Erfahrungsberichte von anderen Fitnesscenter zu finden sein, sowie einige Beispiele zu den umfassenden Statistiken, die man aus GitFit ziehen kann. Ebenfalls soll es inituitiv logisch sein, mit uns in Kontakt zu treten. Auch die Webseite soll ein Gefühl des betreut werden vermitteln.

\subsubsection{Schulungen} \hfill \\
Hat das Produkt im Markt Fuss gefasst, gilt es die Position zu stärken. Dies wird mit motivierten Trainer erledigt, die das Personal des Fitnessstudio für den Einsatz von GitFit schult. Wir bieten 1 tägige Kurse an, an welchem das Personal lern, wie man das volle Potential von GitFit ausnutzen kann.

\subsection{Werbung und Public Relations}
In erster Linie muss ein passendes Fitnesscenter gefunden werden. Dies ist vorzugsweise eine grosse Ketten mit national verteilten Niederlassungen und einem breiten Kundenstamm. Ist ein solcher gefunden, kann über den Verband und Krankenkassen gezielt Werbung für das neue Produkt lanciert werden. Ist das Vertrauen einer grossen Kette einmal gewonnen, ist es wesentlich einfacher dem ganzen Rest unser Produkt vorstellen zu dürfen.

\subsubsection{Angestrebtes Image}
GitFit ist ein innovatives, zukunftsweisendes Startup, das die Fitnessbranche revolutioniert. Fitnesscenter die GitFit unterstützen, werden als modern und attraktiv angesehen. Der Einsatz von GitFit bedeutet für ein Fitnesscenter mehr Kunden, die aus den Möglichkeiten der App, den optimalen Nutzen für ihre Gesundheit ziehen möchten.

\subsubsection{Werbeanstrengungen}
Mit einem prestigeträchtigen Partner, lässt sich Werbung vergleichsweise kostengünstig bewältigen. Die grössten Anstrengungen liegen in der Suche eines solchen Partners. Dazu setzen wir uns mit dem Verband der Schweizerischen Fitness- und Gesundheitscenter (SFGV) und den grössten Fitnesscenter-Ketten zusammen. Diesen soll der Nutzen von GitFit vorgeführt werden. Konnte ein Partner gefunden werden, setzen wir bei unseren Werbeanstrengungen insbesondere auf das Internet. Nachdem die Enwicklung an GitFit abgeschlossen ist, werden wir Werbung in Fachzeitschriften schalten. Diese liegen in den Fitnesscenter meist im Empfangsbereich und werden von wartenden Sportler gerne gelesen.

\subsection{Absatzziele}
Ziel ist es, alle grossen Fitnesscenter-Ketten in der Schweiz mit GitFit auszurüsten. Ist dies geschafft können umfassende Statistiken erstellt werden, die man z.B wieder in die Forschung einfliessen lassen kann und z.B für Krankenkassen interessant sein könnten. Natürlich wäre diese Dienstleistung nicht kostenlos und bedarf einer neuen Evaluierung. Da dies aber ein erfolgreiche Platzierung im Mark voraussetzt, wird dieser Schritt hier nicht mehr weiter erörtert.

\section{Beschaffung und Produktion: die Leistungserstellung}

\section{Management und Organisation: die Köpfe dahinter}

\section{Chancen und Risiken: eine ehrliche Bilanz}

\section{Finanzieller Teil: die nackten Zahlen}

% Doppelte Buchhaltung: heisst aktiva und passiva werden auf zwei verschiedenen Konten verbucht, damit am Schluss der gleiche Betrag herauskommt.
% 
%Bilanz (per Stichtag):
%- Aktiva
%  - Vermögen (Bargeld)
%  - Debitoren (Ausstehende Rechnungen)
%- Materialwert-Abschreibungen
%- Passiva
%  - Kreditor (Schulden)
%  - Aktienkapital
%  - (Gewinn/Verlust)
%-> Bei Aktiva und Passiva muss die gleiche Zahl herauskommen
%
%Erfolgrechne (Periode, zusammenrechnung aus Monaten):
%- Aufwände
%  - Lohnkosten
%  - Miete
%  - Werbung
%
%- Ertrag
%  - Gewinn/Verlust
%  - Verkaufseinnahmen / Dienstleistungen etc.
% Aufwände und Erträge müssen die gleiche Summe haben.


% TODO ZKB KMU Check verwenden
% Das Finanzplanungstool der ZKB muss verwendet werden!
% Investitionsplan für 3 Jahre
% Eröffnungsbilanz + Planbilanzen (3 Jahre) + Planerfolgsrechnungen (3 Jahre) (nur normal case)
% Liquiditätsplan nur für erstes Jahr (pro Monat ausgewiesen)
% Mengengerüst, das den Berechnungen zugrunde liegt.

% TODO Was ist der Kunde bereit zu zahlen?

\section{Umsetzungsplan: die Realisierung}
Musste nicht bearbeitet werden

\section{Latex Syntax}
\subsection{Untersektion}
\subsubsection{Unter Untersektion}
\paragraph{Paragraph (wird eher selten verwendet)} 

\footnote{\url{www.google.ch}}

% Ich bin ein Kommentar
% TODO Ich bin ein TODO

\begin{itemize}
	\item Bullet Point 1 
	\item Bullet Point 2
	\item Bullet Point 3
\end{itemize}

\begin{enumerate}
	\item Zahlen 1
	\item Zahlen 2
\end{enumerate}

\begin{description}
	\item[Begriff] Beschreibung
	\item[Begriff 2] Beschreibung
\end{description}

\paragraph{Mathematische Formel}
\[
	c^2 = \frac{a}{b} \cdot f
\]


\begin{table}[h]
	\centering
	\begin{tabu} to \linewidth {l l l}
		\toprule 
		1 Spalte & 2 Spalte  & 3 Spalte \\
		\midrule
		1 Spalte & 2 Spalte & 3 Spalte \\
		1 Spalte & 2 Spalte & 3 Spalte \\
		\bottomrule 
	\end{tabu} 
	\caption{Ich bin die Beschreibung einer Tabelle}
\end{table}

\begin{table}[h]
	\centering
	\begin{tabu} to \linewidth {l X}
		\toprule 
		1 Spalte  & 3 Spalte \\
		\midrule
		1 Spalte & X sollte verwendet werden, wenn wir sehr sehr sehr sehr sehr sehr sehr sehr sehr sehr sehr sehr sehr sehr sehr sehr sehr sehr sehr sehr sehr lange Zeilen haben \\
		\bottomrule 
	\end{tabu} 
	\caption{Tabelle mit Autmatischen Umbrüchen.}
\end{table}



\clearpage
\appendix

\section{Finanzplan}

Der Finanzplan ist im Anhang \ref{appendix:finanzplan} zu finden.

% List of figures
\listoffigures

% List of tables
\listoftables

% Bibliography
\bibliographystyle{plain} 
\bibliography{literatur}

\label{appendix:finanzplan}
\includepdf[pages={1-},landscape=true]{appendix/finanzplan/finanzplan.pdf}

\end{document}
