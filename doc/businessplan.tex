\documentclass[
a4paper,
oneside,
12pt,
fleqn,
headsepline,
toc=listofnumbered, 
bibliography=totocnumbered]{scrartcl}

% deutsche Trennmuster etc.
\usepackage[T1]{fontenc}
\usepackage[utf8]{inputenc}
\usepackage[english, ngerman]{babel} % \selectlanguage{english} if  needed
\usepackage{lmodern} % use modern latin fonts

% Custom commands
\newcommand{\AUTHOR}{Michael Wieland}
\newcommand{\SECONDAUTHOR}{Fabian Hauser}
\newcommand{\THIRDAUTHOR}{Murièle Trentini}
\newcommand{\FOURTHAUTHOR}{Patrik Scherler}
\newcommand{\INSTITUTE}{Hochschule für Technik Rapperswil}
\newcommand{\LECTURER}{Prof. Dr. Jürg Stadelwieser}
\newcommand{\GITHUB}{https://github.com/michiwieland/businessplan}

% Jede Überschrift 1 auf neuer Seite
\let\stdsection\section
\renewcommand\section{\clearpage\stdsection}

% Multiple Authors
\usepackage{authblk}

% Include external pdf
\usepackage{pdfpages}

% Layout / Seitenränder
\usepackage{geometry}

% Inhaltsverzeichnis
\usepackage{makeidx} 
\makeindex

\usepackage{url}
\usepackage[pdfborder={0 0 0}]{hyperref}
\usepackage[all]{hypcap}
\usepackage{hyperxmp} % for license metadata

% Glossar und Abkürzungsverzeichnis
\usepackage[acronym,toc,nopostdot]{glossaries}
\glossarystyle{altlisthypergroup}
\usepackage{xparse}
\DeclareDocumentCommand{\newdualentry}{ O{} O{} m m m m } {
	\newglossaryentry{gls-#3}{name={#5},text={#5\glsadd{#3}},
		description={#6},#1
	}
	\makeglossaries
	\newacronym[see={[Siehe:]{gls-#3}},#2]{#3}{#4}{#5\glsadd{gls-#3}}
}
\makeglossaries

% Mathematik
\usepackage{amsmath}
\usepackage{amssymb}
\usepackage{amsfonts}
\usepackage{enumitem}

% Images
\usepackage{graphicx}
\graphicspath{{images/}} % default paths

%figures
\usepackage{tikz}
\usetikzlibrary{shapes.geometric}


% Boxes
\usepackage{fancybox}

%Tables
\usepackage{tabu}
\usepackage{booktabs} % toprule, midrule, bottomrule
\usepackage{array} % for matrix tables

% Multi Columns
\usepackage{multicol}

% Header and footer
\usepackage[automark]{scrlayer-scrpage}
\setkomafont{pagehead}{\normalfont}
\setkomafont{pagefoot}{\normalfont}
\automark*{section}
\clearpairofpagestyles
\ihead{\headmark}
\ohead{\TITLE}
\cfoot{\pagemark}
\setlength{\headsep}{30pt}


%Defining the Layer
\makeatletter
\newlength{\topheight}
\setlength{\topheight}{\sls@topmargin}
\addtolength{\topheight}{\headheight}
\DeclareLayer[
background,
contents={%
	\color{\chaptercolor}%
	\rule{\paperwidth}{\topheight}%
}%
]{scrheadings.head.background}
\makeatother

%Adding the Layer to the pagestyles
\AddLayersAtBeginOfPageStyle{scrheadings}{%
	scrheadings.head.background,%
}
\AddLayersAtBeginOfPageStyle{plain.scrheadings}{%
	scrheadings.head.background}

\usepackage{etoolbox}
\newcommand{\chaptercolor}{PrimaryColor}


% Pseudocode
\usepackage{algorithm}
\usepackage{algorithmic}

% Code Listings
\usepackage{listings}
\usepackage{color}
\usepackage{beramono}

\definecolor{DarkPurple}{rgb}{0.4, 0.1, 0.4}
\definecolor{DarkCyan}{rgb}{0.0, 0.5, 0.4}
\definecolor{LightLime}{rgb}{0.3, 0.5, 0.4}
\definecolor{Blue}{rgb}{0.0, 0.0, 1.0}

% Page colors
\definecolor{PrimaryColor}{HTML}{CDDC39}
\definecolor{SecondaryColor}{HTML}{374046}

\usepackage{afterpage}
\usepackage{xcolor}
\usepackage{sectsty}

\color{SecondaryColor}

\lstdefinestyle{eclipse-style}{
	language=Java,
	columns=flexible,
	showstringspaces=false,
	basicstyle=\footnotesize\ttfamily, 
	keywordstyle=\bfseries\color{DarkPurple},
	commentstyle=\color{LightLime},
	stringstyle=\color{Blue}, 
	escapeinside={£}{£}, % latex scope within code
	morekeywords={length},
	numbers=left,
	numberstyle=\tiny\color{black},
	frame=single,
}
\lstset{style=eclipse-style}


%risk-rating
\newcommand\risk[2]{
	\begin{tikzpicture}
	\draw [thick, |->] (0,2) -- (#2,2);
	\draw [fill=PrimaryColor, thick] (#1,2) circle [radius=0.2];
	\end{tikzpicture}
}

% Theorems \begin{mytheo}{title}{label}
\usepackage{tcolorbox}
\tcbuselibrary{theorems}
\newtcbtheorem[number within=section]{definiton}{Definition}%
{fonttitle=\bfseries}{def}
\newtcbtheorem[number within=section]{remember}{Merke}%
{fonttitle=\bfseries}{rem}
\newtcbtheorem[number within=section]{hint}{Hinweis}%
{fonttitle=\bfseries}{hnt}

% Dokumentinformationen
\newcommand{\SUBJECT}{Businessplan}
\newcommand{\TITLE}{GitFit}

% pdf metadata
\hypersetup{
	pdfauthor={\AUTHOR},
	pdftitle={\SUBJECT \TITLE}
}

\begin{document}

% German \and
\renewcommand\Authands{ und }
% Front page
\title{\TITLE}
\subject{\SUBJECT}
\author{\SECONDAUTHOR}
\author{\THIRDAUTHOR}
\author{\FOURTHAUTHOR}
\author{\AUTHOR}
\affil{\INSTITUTE}
\affil{\LECTURER}
\date{\today}
\maketitle

%TODO add front image
\begin{center}
%	\includegraphics[width=0.7\linewidth]{images/front}
\end{center}

\vfill

% Github
\paragraph{Github:} \url{\GITHUB}

% Table of contents
\tableofcontents

% Glossar and acronyms (if included \loadglsentries{glossar})
\printglossary[type=\acronymtype]
\printglossary
\glsaddall


%TODO Total: 20 Seiten
%TODO Gute Idee prüfen: Man verkauft keine Produkte/Dienstleistungen sondern Nutzen! 
	%TODO -> Es muss ersichtlich sein, was das Produkt bringt.
%TODO Zugänglichkeit, Komfort, Image, Lifestyle, Trend

%TODO Buch: Business Model Generation

%TODO Sturktur nach PWC

% Kundenbedürfnisse
%K omfort
%A nsehen
%N euheiten
%S elbsterhaltig
%R isikolos
%O ekonomie (Geld, Zeit)
%O ekologie (Umweltverträglich, BIO)
%S ympathie (menschlich, Design)
%A ngehörigkeit, zugehörigkeit (Teil einer ausgewählten Gruppe)

% Check https://www.ige.ch/

\section{Abstract}
% TODO Remove Example for bibliography entry
\cite{ackema:1998}


% TODO Wir bieten flexibiltät, effizienz, komfort


\section{Das Businessmodell}
%TODO Business Modell Generation = Darstellung in Canvas
% Wert für den Kunden
% Wert für die Firma
% Ändert sich von während der Bearbeitung des Plans
% 9 Bausteine:
% 1. Kundensegmente
% 2. Value Proposition
% 3. Distribution Channels
% 4. Customer Relationships
% 5. Revenue Streams
% 6. Key Ressources
% 7. Key Activities
% 8. Key Partners
% 9. Cost Structure

% Feedback Präsentation:
% - Handys nicht gerne gesehen, Stellenabbau nicht interessant da oft wenige Mitarbeiter, 2 Big Player (Migros, Coop)

\subsection{Kundensegmente}
Betreiber von Fitness Center. 

\subsection{Wertangebote}
KANSROOSA

\subsection{Vertriebskanäle}
App

\subsection{Kundenbeziehungen}

\subsection{Einnahmequellen}

\subsection{Schlüsselressourcen}

\subsection{Schlüsselaktivitäten}


\subsection{Schlüsselpartnerschaften}
Grosse Fitness Center Ketten

\subsection{Kostenstruktur}
Keine Content Kosten. Die Fitnesscenter bauen sich die Stammdaten gegenseitig auf. 

% \includepdf[pages={1},landscape=true]{appendix/schemes/datacenter.pdf}




\section{Mission, Vision und Strategie: die Zukunft}

\section{Produkte und Dienstleistungen: die Marktleistung}

\section{Markt und Kunden: das Zielgebiet}

\section{Konkurrenz: die Mitbewerber}

\section{Marketing: der Weg zum Markt}

\section{Beschaffung und Produktion: die Leistungserstellung}

\section{Management und Organisation: die Köpfe dahinter}

\section{Chancen und Risiken: eine ehrliche Bilanz}

\section{Finanzieller Teil: die nackten Zahlen}
% TODO ZKB KMU Check verwenden
% Das Finanzplanungstool der ZKB muss verwendet werden!
% Investitionsplan für 3 Jahre
% Eröffnungsbilanz + Planbilanzen (3 Jahre) + Planerfolgsrechnungen (3 Jahre) (nur normal case)
% Liquiditätsplan nur für erstes Jahr (pro Monat ausgewiesen)
% Mengengerüst, das den Berechnungen zugrunde liegt.

% TODO Was ist der Kunde bereit zu zahlen?

\section{Umsetzungsplan: die Realisierung}
Musste nicht bearbeitet werden

\clearpage
\appendix

\section{Finanzplan}

Der Finanzplan ist im Anhang \ref{appendix:finanzplan} zu finden.

% List of figures
\listoffigures

% List of tables
\listoftables

% Bibliography
\bibliographystyle{plain} 
\bibliography{literatur}

\label{appendix:finanzplan}
\includepdf[pages={1-},landscape=true]{appendix/finanzplan/finanzplan.pdf}

\end{document}
